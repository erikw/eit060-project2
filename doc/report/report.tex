\documentclass[10pt, a4paper]{article}
\usepackage[T1]{fontenc}
\usepackage[utf8]{inputenc}
\usepackage[swedish]{babel}
\usepackage{ifpdf}
\usepackage[parfill]{parskip}
\usepackage{graphicx}

\title{EIT060 2nd project}
\date{}
\author{
\begin{tabular}{l l}
Tommy Olofsson & \texttt{<ada09tol@student.lu.se>}\\
Erik Westrup & \texttt{<ada09eww@student.lu.se>}\\
Gustaf Waldermarson & \texttt{<ada09gwa@student.lu.se>}\\
Erik Jansson & \texttt{<ada09eja@student.lu.se>}
\end{tabular}
}

\begin{document}

\begin{titlepage}
\maketitle
\begin{center}
  %% stuff in center of titlepage
\end{center}

\thispagestyle{empty}
\end{titlepage}
\setcounter{page}{2}
% content
\section{Description of the system}
The system should handle medical records for patients. It consists of multiple entities including patients, doctors, and agencies. % oxford comma ftw

The entities are as follows:
\begin{itemize}
\item Patient: A patient can have one or more records in the system and he will at all time have read access to them. He will have to use private and public keys and a software client to obtain them from a server. 
\item Doctor: A doctor works at a division and will have read, write and create access for records to all patients registered to him. Also all records associated with the same division will be readable.
\item Nurse: A nurse works under a division will have read and write access to records for records to all patients registered to him/her. Also all records associated with the same division will be readable.
\item Medical record: A medical record is stored on a server. Each record is assosiacted with one patient, one doctor, one nurse and one division.
\item Agency: An agency will have the ability to read and delete records at will.
\item Certificate Authority: All certificates are signed by a root CA and every entety trusts this CA, this CA is
separate from the hospital.
\item The server sits in a locked room and it accessible only by trusted technicans. Is has an audit log containing all access to the patient records.
\end{itemize}

\section{Setup process}
The first step in setting the system up (appart from server installation and hardware distribution) is to generate a keypair and from this a Certificate Signing Request (CSR) for the public key. The CSR is transfered over sneakernet to the CA where it is signed and take back and stored. 

A patient will need to obtain the client software from the ``socialstyrelsen'' office. They will there get their signed certificate stored together with the client on some form of read only memory (e.g compact disc or thumbdrive).

In a further extension of this system, it should be possible to add new doctors and nurses. This, however, is out of the scope of this project.

% TODO expiration of the certs?

\section{Protocol}
\subsection{Read}
Read will return the requested record if the user is 
\begin{itemize}
	\item Patient: A patient can read a specific record with this command by supplying a record id. Returns the record or error if not found.
	\item Doctor, Nurse, Agency: A parameter specifying which record to obtain is needed. Returns the record or an error if not found or access denied.
\end{itemize}

\subsection{Append}
\begin{itemize}
\item Patient: Access always denied with a message "You shall not pass!".
\item Doctor, Nurse: Two parameters with record id and text to append is submitted. Responds with success or denied. Note that a doctor could mistype the id and accidentally write to another patients journal. A final product would include a GUI which would eliminate this problem.
\item Agency: Access always denied with a message "You shall not pass!".
\end{itemize}

\subsection{Create}
Creates a new record for a patient.
\begin{itemize}
\item Patient: Access always denied with a message "You shall not pass!".
\item Doctor: Parameters are patient id, nurse.
\item Nurse: Access always denied with a message "You shall not pass!".
\item Agency: Access always denied with a message "You shall not pass!".
\end{itemize}

\subsection{Delete}
Removes a particular record.
\begin{itemize}
\item Patient: Access always denied with a message "You shall not pass!".
\item Doctor: Access always denied with a message "You shall not pass!".
\item Nurse: Access always denied with a message "You shall not pass!".
\item Agency: Parameter with journal id to delete. Returns success.
\end{itemize}

\subsection{List}
\begin{itemize}
\item Patient: If called by a patient with no parameters, it will list all records pertaining to the patient.
\item Doctor, Nurse, Agency: If called by a doctor or a nurse with a patient id as a parameter, it will list all records belonging to that patient. If no parameter is specified, an error is displayed.
\end{itemize}

\section{Public-key Infrastructure}
\label{sec+pki}

\footnote{Computer Security 3ed, Dieter Gollman, Wiley, page 288-293}

\section{SSL/TLS}
In our system we are using \emph{SSL/TLS}\footnote{Secure Socket Layer/Transport Layer Security}-technology for authentication and establishment of a secure connection between the communicating parties. SSL was developed as a client-server protocol to be used to secure connections above the transport layer in the TCP/IP-stack for application protocols like http and email. Work on SSL started in the mid 90s by Netscape and was eventually adopted as a standard. SSL has lately been superseded by TLS. However the last version of SSL, 3.0, only differs slightly from TLS 1.0 and thus you often refer to the technology as SSL/TLS. We will here refer to TLS as it's the new  and recommended protocol to use. By using TLS in our program we will achieve mutual authentication between server and client, message confidentiality and integrity. The mutual two-way authentication is optional but since w are using it it will be described below.

\subsection{Pre-handshake}
It is assumed that both the server and the client have generated a public/private key pair and have a CSR signed a CA according to section \ref{sec+pki}. It is essential that both sides trust at least one CA in the certificate chain of the other party -- otherwise they will not be able to trust each other.

\subsection{Handshake}
\subsection{Communication}
\subsection{Teardown}


\footnote{Computer Security 3ed, Dieter Gollman, Wiley, page 310-314}
\footnote{Java\textregistered Secure Socket Extension (JSSE) Reference Guide, accessed 2012-02-09, http://docs.oracle.com/javase/6/docs/technotes/guides/security/jsse/JSSERefGuide.html}






\end{document}
