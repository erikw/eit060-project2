\documentclass[10pt, a4paper]{article}
\usepackage[T1]{fontenc}
\usepackage[utf8]{inputenc}
\usepackage[swedish]{babel}
\usepackage{ifpdf}
\usepackage[parfill]{parskip}
\usepackage{graphicx}

\title{EIT060 2nd project}
\date{}
\author{
	\begin{tabular}{l l}
		Tommy Olofsson & \texttt{<ada09tol@student.lu.se>}\\
		Erik Westrup & \texttt{<ada09eww@student.lu.se>}\\
		Gustaf Waldermarson & \texttt{<ada09gwa@student.lu.se>}\\
		Erik Jansson & \texttt{<ada09eja@student.lu.se>}
	\end{tabular}
}

\begin{document}

\begin{titlepage}
\maketitle
\begin{center}
  %% stuff in center of titlepage
\end{center}

\thispagestyle{empty}
\end{titlepage}
\setcounter{page}{2}
% content
\section{Description of the system}
The system should handle medical records for patients. It consists of multiple entities including patients, doctors, and agencies. % oxford comma ftw

The entities are as follows:
\begin{itemize}
\item Patient: A patient can have one or more records in the system and he will at all time have read access to them. He will have to use private and public keys and a software client to obtain them from a server. 
\item Doctor: A doctor works at a division and will have read, write and create access for records to all patients registered to him. Also all records associated with the same division will be readable.
\item Nurse: A nurse works under a division will have read and write access to records for records to all patients registered to him/her. Also all records associated with the same division will be readable.
\item Medical record: A medical record is stored on a server. Each record is assosiacted with one patient, one doctor, one nurse and one division.
\item Agency: An agency will have the ability to read and delete records at will.
\item Certificate Authority: All certificates are signed by a root CA and every entety trusts this CA, this CA is
separate from the hospital.
\item The server sits in a locked room and it accessible only by trusted technicans. Is has an audit log containing all access to the patient records.
\end{itemize}

\section{Setup process}
The first step in setting the system up (appart from server installation and hardware distribution) is to generate a keypair and from this a Certificate Signing Request (CSR) for the public key. The CSR is transfered over sneakernet to the CA where it is signed and take back and stored. 

A patient will need to obtain the client software from the ``socialstyrelsen'' office. They will there get their signed certificate stored together with the client on some form of read only memory (e.g compact disc or thumbdrive).

In a further extension of this system, it should be possible to add new doctors and nurses. This, however, is out of the scope of this project.

% TODO expiration of the certs?

\section{Protocol}
\subsection{Read}
Read will return the requested record if the user is 
\begin{itemize}
	\item Patient: A patient can read a specific record with this command by supplying a record id. Returns the record or error if not found.
	\item Doctor, Nurse, Agency: A parameter specifying which record to obtain is needed. Returns the record or an error if not found or access denied.
\end{itemize}

\subsection{Append}
\begin{itemize}
\item Patient: Access always denied with a message "You shall not pass!".
\item Doctor, Nurse: Two parameters with record id and text to append is submitted. Responds with success or denied. Note that a doctor could mistype the id and accidentally write to another patients journal. A final product would include a GUI which would eliminate this problem.
\item Agency: Access always denied with a message "You shall not pass!".
\end{itemize}

\subsection{Create}
Creates a new record for a patient.
\begin{itemize}
\item Patient: Access always denied with a message "You shall not pass!".
\item Doctor: Parameters are patient id, nurse.
\item Nurse: Access always denied with a message "You shall not pass!".
\item Agency: Access always denied with a message "You shall not pass!".
\end{itemize}

\subsection{Delete}
Removes a particular record.
\begin{itemize}
\item Patient: Access always denied with a message "You shall not pass!".
\item Doctor: Access always denied with a message "You shall not pass!".
\item Nurse: Access always denied with a message "You shall not pass!".
\item Agency: Parameter with journal id to delete. Returns success.
\end{itemize}

\subsection{List}
\begin{itemize}
\item Patient: If called by a patient with no parameters, it will list all records pertaining to the patient.
\item Doctor, Nurse, Agency: If called by a doctor or a nurse with a patient id as a parameter, it will list all records belonging to that patient. If no parameter is specified, an error is displayed.
\end{itemize}

\subsection{Implementation}
In the name of simplicity and convenience for the implementators no fancy protocol is used. The same commands describes above are sent directly to the server from the client with a message length perpended. The server can answer with to commands: error and result. The first is sent when a requested action is denied or failed and a text message is included. If the requested was successful a result is sent with the result text included.

\section{Public-key Infrastructure}
\label{sec+pki}

\footnote{Computer Security 3ed, Dieter Gollman, Wiley, page 288-293}

\section{SSL/TLS}
\label{sec+tls}
In our system we are using \emph{SSL/TLS}\footnote{Secure Socket Layer/Transport Layer Security}-technology for authentication and establishment of a secure connection between the communicating parties. SSL was developed as a client-server protocol to be used to provide a secure byte stream on top of transport layer in the TCP/IP-stack for application protocols like http and email. Work on SSL started in the mid 90s by Netscape and was eventually adopted as a standard. SSL has lately been superseded by TLS. However the last version of SSL, 3.0, only differs slightly from TLS 1.0 and thus you often refer to the technology as SSL/TLS. We will here refer to TLS as it's the new  and recommended protocol to use. By using TLS in our program we will achieve mutual authentication between server and client, message confidentiality and integrity. The mutual two-way authentication is optional but since we are using it it will be described below.


\subsection{Pre-handshake}
It is assumed that both the server and the client have generated a public/private key pair and have a CSR signed a CA according to section \ref{sec+pki}. It is essential that both sides trust at least one CA in the certificate chain of the other party -- otherwise they will not be able to trust each other.

\subsection{Handshake}
Before the parties can begin setting up the parameters the client must identify the server so it knows to whom he is about to request services from. The server must identify the client to be able to correctly authenticate it so it can use the provided services. They might not have had a previous connection so that they don't have a shared secret. This is solved by letting a trusted third party\footnote{TTP} verify the identity of one part. Thus the first thing to do is to exchange certificates 

\subsection{Communication}
\subsection{Teardown}


\footnote{Computer Security 3ed, Dieter Gollman, Wiley, page 310-314}
\footnote{Java\textregistered Secure Socket Extension (JSSE) Reference Guide, accessed 2012-02-09, http://docs.oracle.com/javase/6/docs/technotes/guides/security/jsse/JSSERefGuide.html}


\section{Security evaluation}
A list of attacks we are safe against:
\begin{description}
\item[Man-in-the-middle] Defending against a man-in-the-middle attack is hard without using a second, secure channel for verification. By using SSL-connections and certificates signed by a common certificate authority, this attack is not possible. An eavesdropper, from here on called Eve, would need to trick the certificate authority into signing her certificate with the servers name on it. This should not be possible since the CA checks identities of all certificates it signs.
\item[Eavesdropping] This attack is simply averted by using the SSL protocol, see Section~\ref{sec+tls} for a more detailed explanation of SSL.
\item[Fake certificates] Since we're using a single certificate authority which signs all the certificates, no certificates signed by anyone else than our CA is accepted. Thus, this attack vector is eliminated.
\end{description}

A list of attacks we are vulnerable against:
\begin{description}
\item[Spoofed client] If an attacker writes a fake client and tricks a user into using it as the real one, we cannot defend against that. The most severe thing the malicious attacker can do in this scenario is to steal all health information for that patient. Unfortunately, the only way to defend against this is to make sure that the correct client has been downloaded by visiting the hospital website using an SSL-connection.
\item[]
\end{description}

The human factor is hard to take into account, a user will always do things you have not predicted with the software. For example, our implementation cannot protect against the user choosing a short and weak password which are prone to be broken since the password is handled by the Java keytool utility.

In a real system, the client software would put some requirements on the password, e.g.~alphanumerical passwords of length 10 or more. This would give $(26*2 + 10)^{10} \approx 8.39\cdot 10^{17}$ possible combination of passwords which would take approximately 2 years and 7 months to enumerate at 10 billion passwords per second\footnote{http://www.wolframalpha.com/input/?i=\%2826*2\%2B10\%29\%5E10+\%2F+10\%5E10+s}. Since we will have a expiration time of one year, this seems sufficient protection against weak passwords.

Some user education would be required to make sure they always download the software from the hospitals secure web server which authenticates itself with an SSL-certificate signed by a certificate authority which is trusted by all browsers, e.g.~VeriSign or DigiCert. This is needed to avoid the possibility that the user can accidentally receive a malicious client which steals patient information.

\end{document}
